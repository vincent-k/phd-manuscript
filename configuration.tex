%\usepackage{vmargin}
%\special{papersize=160mm,240mm}
%\setpapersize{custom}{160mm}{240mm}
%\setmargins{1.2cm}{1.5cm}{13cm}{18.5cm}{0.6cm}{0.8cm}{0cm}{1.0cm}

% algpseudocode settings
%\algrenewcommand\alglinenumber[1]{
%    {\footnotesize#1}
%}

% prevent footnotes from breaking across pages
\interfootnotelinepenalty=10000

% redefine algorithms' frame
\makeatletter
\renewcommand\fs@ruled{%
  \def\@fs@cfont{\rmfamily}%
  \let\@fs@capt\floatc@plain%
  \def\@fs@pre{\hrule height.8pt depth0pt \kern2pt}%% or use \def\@fs@pre{} to get rid of the top rule
  \def\@fs@post{}%\def\@fs@post{\kern2pt\hrule\relax}%% or use \def\@fs@post{} to get rid of the last rule
  \def\@fs@mid{\kern2pt\hrule\kern15pt}%% or use \def\@fs@mid{} to get rid of the middle rule
  \let\@fs@iftopcapt\iffalse}
\makeatother

% define new algorithmicx blocks
\algblock{ForEach}{EndForEach}
\algnewcommand\algorithmicforeach{\textbf{for\ each}}
\algnewcommand\algorithmicforeachdo{\textbf{do}}
\algnewcommand\algorithmicendforeach{\textbf{end\ for\ each}}
\algrenewtext{ForEach}[1]{\algorithmicforeach\ #1\ \algorithmicforeachdo}
\algrenewtext{EndForEach}{\algorithmicendforeach}

\algblock{ForEachParallel}{EndForEachParallel}
\algnewcommand\algorithmicforeachparallel{\textbf{for\ each}}
\algnewcommand\algorithmicforeachparalleldo{\textbf{do\ in\ parallel}}
\algnewcommand\algorithmicendforeachparallel{\textbf{end\ for}}
\algrenewtext{ForEachParallel}[1]{\algorithmicforeachparallel\ #1\ \algorithmicforeachparalleldo}
\algrenewtext{EndForEachParallel}{\algorithmicendforeachparallel}

\algnewcommand{\CommentLeft}[1]{\State \(\triangleright\) #1}

% hyperref configuration
\hypersetup{
    colorlinks=false,
    linkcolor=red,
    urlcolor=blue,
    citecolor=green
}

% custom colors
\definecolor{darkgreen}{rgb}{0,0.6,0}

% listing configuration

\lstdefinestyle{java} {
    aboveskip=1.5em,
    basicstyle=\ttfamily\footnotesize,
    captionpos=b,
    frame=single, 
    identifierstyle=,
    language=Java,
    tabsize=4,
    numbers=left, 
    numberstyle=\scriptsize,
    xleftmargin=1.15em,     
    framexleftmargin=2em,
    escapeinside={\$}{\$},
	%commentstyle=\color{gray}
}

\lstdefinestyle{plain} {
    aboveskip=1.5em,
    basicstyle=\ttfamily\footnotesize,
    captionpos=b,
    frame=single, 
    identifierstyle=
}

\lstdefinestyle{python} {
    aboveskip=1.5em,
    language=Python,
    showstringspaces=false,
    formfeed=\newpage,
    tabsize=4,
    commentstyle=\itshape,
    captionpos=b,
    basicstyle=\ttfamily\footnotesize,
    numbers=left, 
    numberstyle=\scriptsize, 
    frame=single, 
    xleftmargin=1.15em,     
    framexleftmargin=2em,
    escapeinside={@}{@}
}

\lstdefinestyle{sparql} {
    aboveskip=1.5em,
    language=SPARQL,
    showstringspaces=false,
    formfeed=\newpage,
    tabsize=4,
    commentstyle=\itshape,
    captionpos=b,
    basicstyle=\ttfamily\footnotesize,
    numbers=left, 
    numberstyle=\scriptsize, 
    frame=single, 
    xleftmargin=1.15em,     
    framexleftmargin=2em,
    escapeinside={@}{@}
}

\lstdefinestyle{rdf} {
    aboveskip=1.5em,
    showstringspaces=false,
    formfeed=\newpage,
    tabsize=4,
    commentstyle=\itshape,
    captionpos=b,
    basicstyle=\ttfamily\footnotesize,
    numbers=left, 
    numberstyle=\scriptsize, 
    frame=single, 
    xleftmargin=1.15em,     
    framexleftmargin=2em,
    escapeinside={@}{@}
}

% tikz configuration
\usetikzlibrary{
    arrows,
    backgrounds,
    calc,
    chains,
    decorations.markings,
    decorations.pathmorphing,
    fit,
    matrix,
    patterns,
    plotmarks,
    positioning,
    shadows,
    shapes
}

% new PGF shapes

% t shape for component interfaces
\makeatletter
\pgfdeclareshape{t}{
    \inheritsavedanchors[from=rectangle]
    \inheritanchorborder[from=rectangle]
    \inheritanchor[from=rectangle]{center}
    \inheritanchor[from=rectangle]{base}
    \inheritanchor[from=rectangle]{north}
    \inheritanchor[from=rectangle]{north east}
    \inheritanchor[from=rectangle]{east}
    \inheritanchor[from=rectangle]{south east}
    \inheritanchor[from=rectangle]{south}
    \inheritanchor[from=rectangle]{south west}
    \inheritanchor[from=rectangle]{west}
    \inheritanchor[from=rectangle]{north west}
    \backgroundpath{
        % store lower right in xa/ya and upper right in xb/yb
       \southwest \pgf@xa=\pgf@x \pgf@ya=\pgf@y
       \northeast \pgf@xb=\pgf@x \pgf@yb=\pgf@y
       \pgfmathparse{(\pgf@xb-\pgf@xa)/2}
       \pgf@xc=\pgf@xa
       \advance\pgf@xc by \pgfmathresult pt
       \pgfpathmoveto{\pgfpoint{\pgf@xa}{\pgf@ya}}
       \pgfpathmoveto{\pgfpoint{\pgf@xa}{\pgf@yb}}
       \pgfpathlineto{\pgfpoint{\pgf@xb}{\pgf@yb}}
       \pgfpathmoveto{\pgfpoint{\pgf@xc}{\pgf@yb}}
       \pgfpathlineto{\pgfpoint{\pgf@xc}{\pgf@ya}}
   }
}
\makeatother

\captionsetup[algorithm]{
    justification=centering,labelfont=normalfont,labelsep=endash
}

% increase line spacing
\onehalfspacing

% rename the default name of the table of contents
% from 'Contents' to 'Table of Content'
\addto\captionsenglish{
    \renewcommand\contentsname{Table of Contents\vspace{-5pt}}
}

% define abstract environment since it is not available 
% with book document class
\newenvironment{abstract}{
    \null\vfill
    \begin{center}
        \bfseries\abstractname
    \end{center}
}{
    \vfill\null\clearpage
}

% custom minitoc command to have extra space after 
% the mini table of content
\newcommand{\cminitoc}{
    \minitoc
    \vspace{1em}
}

% Group list of listings by chapter
\makeatletter
\let\my@chapter\@chapter
\renewcommand*{\@chapter}{%
  \addtocontents{lol}{\protect\addvspace{7pt}}%
  \my@chapter}
\makeatother

% reconfigure list of listings commands
\renewcommand{\lstlistlistingname}{List of Listings}

\renewcommand{\lstlistoflistings}{
    \begingroup
    \tocfile{\lstlistlistingname}{lol}
    \endgroup
    \mtcaddchapter
}

% tabular helper
\newcommand*{\tabbox}[2][t]{
        \vspace{0pt
    }\parbox[#1][3.7\baselineskip]{1.35cm}{\strut#2\strut}
}

% bibliography, backref text
%\DefineBibliographyStrings{english}{
 %   backrefpage = {cited on p.},
 %   backrefpages= {cited on pp.},
%}

\newcolumntype{Y}{>{\centering\arraybackslash}X}

\definecolor{orange}{rgb}{1,0.5,0}
\definecolor{redcomment}{rgb}{0.5,0,0}


\lstdefinelanguage{Coqfix}{  
  sensitive=true,
  morecomment=[l]{//},      %single line comment 
  morecomment=[s]{(*}{*)}, %normal comment
 morecomment=[n]{(*}{*)},  %nested comment
  morestring=[b]",    
  basicstyle=\footnotesize,           % the size of the fonts that are used for the code
  numbers=left,                   % where to put the line-numbers
  numberstyle=\tiny\color{black},  % the style that is used for the line-numbers
  stepnumber=1,                   % the step between two line-numbers. If it's 1, each line 
                                  % will be numbered
  numbersep=5pt,                  % how far the line-numbers are from the code
  backgroundcolor=\color{white},      % choose the background color. You must add \usepackage{color}
  showspaces=false,               % show spaces adding particular underscores
  showstringspaces=false,         % underline spaces within strings
  showtabs=false,                 % show tabs within strings adding particular underscores
  %frame=single,                   % adds a frame around the code
  rulecolor=\color{black},        % if not set, the frame-color may be changed on line-breaks within not-black text (e.g. commens (green here))
  tabsize=2,                      % sets default tabsize to 2 spaces
  captionpos=b,                   % sets the caption-position to bottom
  breaklines=true,                % sets automatic line breaking
  breakatwhitespace=false,        % sets if automatic breaks should only happen at whitespace
  %title=\lstname,                   % show the filename of files included with \lstinputlisting;
                                  % also try caption instead of title
  keywordstyle=\color{red},          % keyword style
  keywordstyle=[2]\color{blue},          % keyword style
  commentstyle=\color{redcomment},       % comment style
  stringstyle=\color{orange},         % string literal style
  escapeinside={\%*}{*)},            % if you want to add a comment within your code
  morekeywords={Inductive, Definition, Function, Fixpoint, Lemma, Theorem, Eval, Parameter},      % if you want to add more keywords to the set
 morekeywords={[2] Prop, Type, with, match, end, Proof, Qed, forall, Notation, if, then, else, Case, compute, let, in, return, Defined, as, fix}   % if you want to add more keywords to the set
}

\lstdefinelanguage{Coq}{  
  sensitive=true,
  morecomment=[l]{//},      %single line comment 
  morecomment=[s]{(*}{*)}, %normal comment
 morecomment=[n]{(*}{*)},  %nested comment
  morestring=[b]",    
  basicstyle=\footnotesize,           % the size of the fonts that are used for the code
  numbers=left,                   % where to put the line-numbers
  numberstyle=\tiny\color{black},  % the style that is used for the line-numbers
  stepnumber=1,                   % the step between two line-numbers. If it's 1, each line 
                                  % will be numbered
  numbersep=5pt,                  % how far the line-numbers are from the code
  backgroundcolor=\color{white},      % choose the background color. You must add \usepackage{color}
  showspaces=false,               % show spaces adding particular underscores
  showstringspaces=false,         % underline spaces within strings
  showtabs=false,                 % show tabs within strings adding particular underscores
  %frame=single,                   % adds a frame around the code
  rulecolor=\color{black},        % if not set, the frame-color may be changed on line-breaks within not-black text (e.g. commens (green here))
  tabsize=2,                      % sets default tabsize to 2 spaces
  captionpos=b,                   % sets the caption-position to bottom
  breaklines=true,                % sets automatic line breaking
  breakatwhitespace=false,        % sets if automatic breaks should only happen at whitespace
  %title=\lstname,                   % show the filename of files included with \lstinputlisting;
                                  % also try caption instead of title
  keywordstyle=\color{red},          % keyword style
  keywordstyle=[2]\color{blue},          % keyword style
  commentstyle=\color{redcomment},       % comment style
  stringstyle=\color{orange},         % string literal style
  escapeinside={\%*}{*)},            % if you want to add a comment within your code
  morekeywords={Inductive, Definition, Function, Fixpoint, Lemma, Theorem, Eval, Parameter},      % if you want to add more keywords to the set
 morekeywords={[2] Prop, Type, with, match, end, Proof, Qed, forall, Notation, if, then, else, Case, compute, let, in, return, Defined, as, exists}   % if you want to add more keywords to the set
}


\lstdefinelanguage{OCaml}{  
  sensitive=true,
  morecomment=[l]{//},      %single line comment 
  morecomment=[s]{(*}{*)}, %normal comment
 morecomment=[n]{(*}{*)},  %nested comment
  morestring=[b]",    
  basicstyle=\footnotesize,           % the size of the fonts that are used for the code
  numbers=left,                   % where to put the line-numbers
  numberstyle=\tiny\color{black},  % the style that is used for the line-numbers
  stepnumber=1,                   % the step between two line-numbers. If it's 1, each line 
                                  % will be numbered
  numbersep=5pt,                  % how far the line-numbers are from the code
  backgroundcolor=\color{white},      % choose the background color. You must add \usepackage{color}
  showspaces=false,               % show spaces adding particular underscores
  showstringspaces=false,         % underline spaces within strings
  showtabs=false,                 % show tabs within strings adding particular underscores
  %frame=single,                   % adds a frame around the code
  rulecolor=\color{black},        % if not set, the frame-color may be changed on line-breaks within not-black text (e.g. commens (green here))
  tabsize=2,                      % sets default tabsize to 2 spaces
  captionpos=b,                   % sets the caption-position to bottom
  breaklines=true,                % sets automatic line breaking
  breakatwhitespace=false,        % sets if automatic breaks should only happen at whitespace
  %title=\lstname,                   % show the filename of files included with \lstinputlisting;
                                  % also try caption instead of title
  keywordstyle=\color{red},          % keyword style
  keywordstyle=[2]\color{blue},          % keyword style
  commentstyle=\color{redcomment},       % comment style
  stringstyle=\color{black},         % string literal style
  escapeinside={\%*}{*)},            % if you want to add a comment within your code
  morekeywords={ int, float, string, Obj},               % if you want to add more keywords to the set
 morekeywords={[2] fun, let, function, if, then, else, match, with, in}               % if you want to add more keywords to the set
}


\lstdefinelanguage{MyXML}{  
  sensitive=true,
  %morecomment=[l]{<!--},      %single line comment 
  morecomment=[s]{<!--}{-->}, %normal comment
 morecomment=[n]{<!--}{-->},  %nested comment
  morestring=[b]",    
  basicstyle=\footnotesize,           % the size of the fonts that are used for the code
  numbers=left,                   % where to put the line-numbers
  numberstyle=\tiny\color{black},  % the style that is used for the line-numbers
  stepnumber=1,                   % the step between two line-numbers. If it's 1, each line 
                                  % will be numbered
  numbersep=5pt,                  % how far the line-numbers are from the code
  backgroundcolor=\color{white},      % choose the background color. You must add \usepackage{color}
  showspaces=false,               % show spaces adding particular underscores
  showstringspaces=false,         % underline spaces within strings
  showtabs=false,                 % show tabs within strings adding particular underscores
  %frame=single,                   % adds a frame around the code
  rulecolor=\color{black},        % if not set, the frame-color may be changed on line-breaks within not-black text (e.g. commens (green here))
  tabsize=2,                      % sets default tabsize to 2 spaces
  captionpos=b,                   % sets the caption-position to bottom
  breaklines=true,                % sets automatic line breaking
  breakatwhitespace=false,        % sets if automatic breaks should only happen at whitespace
  %title=\lstname,                   % show the filename of files included with \lstinputlisting;
                                  % also try caption instead of title
  keywordstyle=\color{red},          % keyword style
  keywordstyle=[2]\color{blue},          % keyword style
  keywordstyle=[3]\bfseries,
  keywordstyle=[4]\color{darkgreen},
  commentstyle=\color{redcomment},       % comment style
  stringstyle=\color{orange},         % string literal style
  escapeinside={\%*}{*)},            % if you want to add a comment within your code
  morekeywords={DOCTYPE, PUBLIC},            
 morekeywords={[2] definition},       
 morekeywords={[3] interface, content, controller},
 morekeywords={[4] desc, class, signature, name, role}
}

\lstdefinelanguage{Fiacre}{  
  sensitive=true,
  morecomment=[l]{//},      %single line comment 
  morecomment=[s]{/*}{*/}, %normal comment
 morecomment=[n]{/*}{*/},  %nested comment
  morestring=[b]",    
  basicstyle=\footnotesize,           % the size of the fonts that are used for the code
  numbers=left,                   % where to put the line-numbers
  numberstyle=\tiny\color{black},  % the style that is used for the line-numbers
  stepnumber=1,                   % the step between two line-numbers. If it's 1, each line 
                                  % will be numbered
  numbersep=5pt,                  % how far the line-numbers are from the code
  backgroundcolor=\color{white},      % choose the background color. You must add \usepackage{color}
  showspaces=false,               % show spaces adding particular underscores
  showstringspaces=false,         % underline spaces within strings
  showtabs=false,                 % show tabs within strings adding particular underscores
  %frame=single,                   % adds a frame around the code
  rulecolor=\color{black},        % if not set, the frame-color may be changed on line-breaks within not-black text (e.g. commens (green here))
  tabsize=2,                      % sets default tabsize to 2 spaces
  captionpos=b,                   % sets the caption-position to bottom
  breaklines=true,                % sets automatic line breaking
  breakatwhitespace=false,        % sets if automatic breaks should only happen at whitespace
  %title=\lstname,                   % show the filename of files included with \lstinputlisting;
                                  % also try caption instead of title
  keywordstyle=\color{red},          % keyword style
  keywordstyle=[2]\color{blue},          % keyword style
  keywordstyle=[3]\bfseries,
  keywordstyle=[4]\color{darkgreen},
  commentstyle=\color{redcomment},       % comment style
  stringstyle=\color{orange},         % string literal style
  escapeinside={\%*}{*)},            % if you want to add a comment within your code
  morekeywords={type, process, union, is, record, const, component, par},            
 morekeywords={[2] from, if, to, end, select},       
 morekeywords={[3] var, states},
 morekeywords={[4] in, out}
}

\lstdefinelanguage{Fiacretypes}{  
  sensitive=true,
  morecomment=[l]{//},      %single line comment 
  morecomment=[s]{/*}{*/}, %normal comment
 morecomment=[n]{/*}{*/},  %nested comment
  morestring=[b]",    
  basicstyle=\footnotesize,           % the size of the fonts that are used for the code
  numbers=left,                   % where to put the line-numbers
  numberstyle=\tiny\color{black},  % the style that is used for the line-numbers
  stepnumber=1,                   % the step between two line-numbers. If it's 1, each line 
                                  % will be numbered
  numbersep=5pt,                  % how far the line-numbers are from the code
  backgroundcolor=\color{white},      % choose the background color. You must add \usepackage{color}
  showspaces=false,               % show spaces adding particular underscores
  showstringspaces=false,         % underline spaces within strings
  showtabs=false,                 % show tabs within strings adding particular underscores
  %frame=single,                   % adds a frame around the code
  rulecolor=\color{black},        % if not set, the frame-color may be changed on line-breaks within not-black text (e.g. commens (green here))
  tabsize=2,                      % sets default tabsize to 2 spaces
  captionpos=b,                   % sets the caption-position to bottom
  breaklines=true,                % sets automatic line breaking
  breakatwhitespace=false,        % sets if automatic breaks should only happen at whitespace
  %title=\lstname,                   % show the filename of files included with \lstinputlisting;
                                  % also try caption instead of title
  keywordstyle=\color{red},          % keyword style
  keywordstyle=[2]\color{blue},          % keyword style
  keywordstyle=[3]\bfseries,
  keywordstyle=[4]\color{darkgreen},
  commentstyle=\color{redcomment},       % comment style
  stringstyle=\color{orange},         % string literal style
  escapeinside={\%*}{*)},            % if you want to add a comment within your code
  morekeywords={type, process, union, is, record, const, component, par, end},            
 morekeywords={[2] from, if, to, select},       
 morekeywords={[3] var, states},
 morekeywords={[4] in, out}
}