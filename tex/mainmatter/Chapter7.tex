% Chapter 7

\chapter{Final remarks} 
\label{chap:conclusion} 

\epigraph{\textit{“Success is not final, failure is not fatal: it is the courage to continue that counts."}}{Winston Churchill}



\minitoc


\lhead{Chapter 8. \emph{Final Remarks}} % This is for the header on each page - perhaps a shortened title

%----------------------------------------------------------------------------------------


		This thesis discussed the application of formal methods techniques in the context of 
	component-based engineering. In particular, we focused on the \ac{GCM} and its reference 
	implementation GCM/ProActive.    	

		Our industrial case study about 
	\textsc{The HyperManager} application demonstrated the pragmatic rationale of
	formal verification by means of model-checking techniques. Indeed, the degree of automation
	offered by this approach makes it appealing for the purposes of increasing the confidence
	one can put on the properties of a verified software system. However, it is also fair to say that,
	by itself, remains a rather hopeless approach for verifying parametrized systems dealing
	with infinite data structures. Indeed, one can try to define finite abstractions of 
	the system to model. However, then one needs to prove that these abstractions are sound 
	w.r.t. the original system semantics, and that is seldom a trivial task.
	
		To this end, we developed \textsc{Mefresa}, a mechanized framework for reasoning on software 
	architectures. Behavioural concerns are addressed by a formalization of
	a semantics based on execution traces of synchronized transition systems.
	This, provides the first steps towards a complete specification and verification
	platform addressing both architectural and behavioural properties. 
	
		Moreover, we further leverage \textsc{Mefresa} by using its \textsf{operation} language
	as a target for the evaluation of \textsc{Painless} specifications. Using Coq's extraction
	mechanism we get certified code checking well-formedness and well-typedness compliance.	
	In conjunction with a \textsc{Painless} interpreter, it is subsequently integrated with 
	ProActive, yielding a library for the safe deployment and reconfiguration of
	 \ac{GCM} applications.
		
		
	To conclude, we use Coq to address the specification, verification and deployment of
	distributed and reconfigurable \ac{GCM} applications. 

