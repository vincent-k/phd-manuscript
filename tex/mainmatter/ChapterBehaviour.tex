% Chapter 7

\chapter{Mechanized behavioural semantics} 
\label{chap:behaviour} 

\epigraph{\textit{“Yet, ..."}}{Nuno Gaspar}



\minitoc


\lhead{Chapter 7. \emph{Mechanized behavioural semantics}} % This is for the header on each page - perhaps a shortened title

%----------------------------------------------------------------------------------------

	This chapter discusses the mechanization, in the Coq proof assistant, of a behavioural 
semantics based on the execution trace of synchronized labelled transition systems. Further, we show
how it can be used in the context of \ac{GCM} applications.

	Section \ref{sec:pLTS} presents the mechanization of \ac{LTS} and
their traces. We show how we can synchronize several \ac{LTS} in Section \ref{sec:pnet}.
Then, we exemplify its use in the context of \ac{GCM} applications in Section \ref{sec:gcmpnets}.
For last, Section \ref{sec:behaviourdiscussion} discusses the final remarks about this
mechanization.



\section{Groundwork: infinite objects and LTL}
\label{sec:groundwork}

	\cite[chap. 13]{opac-b1101046}

\subsection{Lazy lists}
\label{sub:lazy}


\subsection{Linear temporal logic}
\label{sub:ltl}


\section{Labelled transition systems, and traces}
\label{sec:pLTS}

	A LTS is composed by a set of states, an initial state, a set of actions, and a set of transitions, each composed
	by a source state, a transition, and a target state. First, let us see the datatype used to encode
	a LTS state. Listing \ref{lst:ltsstate} depicts such datatype.
	
	\lstinputlisting[language=Coq, stepnumber=1, 
	                      caption={\textsf{lts\_state} datatype}, 
	                      %firstnumber=19,
	                      label=lst:ltsstate]{listings/chapter6/state.tex}	
	
	
	\noindent A \textsf{lts\_state} is identified by a natural number, and holds an internal memory that is a
	simple mapping between strings to naturals. For the sake of simplicity we directly use strings to denote variables,
	and constrain ourselves to natural number values.

	Next, let us now see the \textsf{action} datatype. Listing \ref{lst:action} illustrates its definition.
	
		\lstinputlisting[language=Coq, stepnumber=1, 
	                      caption={\textsf{action} datatype}, 
	                      %firstnumber=19,
	                      label=lst:action]{listings/chapter6/action.tex}	

	\noindent An \textsf{action} is composed by a \textsf{message}, and a set of \textsf{assignments}.
	As expected, the latter permits to specify the assignment of specific state variables to a \textsf{transition}.
	The former holds the label, its type of communication and a set of parameters. Listing \ref{lst:message}
	depicts its definition.
	
			\lstinputlisting[language=Coq, stepnumber=1, 
	                      caption={\textsf{message} datatype}, 
	                      %firstnumber=19,
	                      label=lst:message]{listings/chapter6/message.tex}	

	\noindent A message can either be \textit{reading} or \textit{emitting} (lines 2-3). Typically, these
	are used to receive and transmit values, respectively. Naturally, the allowed \textsf{parameters} are
	either values --- natural numbers --- or variables (line 6-7).
	
	
	Finally, we can now see the \textsf{LTS} datatype as depicted by Listing \ref{lst:lts}.	
		
	\lstinputlisting[language=Coq, stepnumber=1, 
	                      caption={\textsf{LTS} datatype}, 
	                      %firstnumber=19,
	                      label=lst:lts]{listings/chapter6/LTS.tex}	

	Having defined the \textsf{LTS} datatype, we can now define \textsf{traces}. A trace is a sequence of actions 
	resulting from the sequence of \textsf{transitions} taken by the \textsf{LTS}. It can either be finite or
	infinite. A finite trace means that a \textit{sink} state was attained, thus the execution \textit{deadlocks}.
	Listing \ref{lst:ltstrace} defines a \textsf{LTS} trace.	
	
	
		\lstinputlisting[language=Coq, stepnumber=1, 
	                      caption={Trace definition for a \textsf{LTS}}, 
	                      %firstnumber=19,
	                      label=lst:ltstrace]{listings/chapter6/ltstrace.tex}	


	\noindent A \textsf{LTS\_Trace} is defined by a co-inductive predicate since we potentially deal
	with infinite sequences. The expression, \textsf{LTS\_Trace A q l}, means that \textsf{l} is a trace
	in the \textsf{LTS} object \textsf{A}, starting from the state \textsf{q}.   

		The \textsf{LTS\_Trace} predicate is composed by two constructors: \textsf{lts\_empty\_trace} (line 2)
	and \textsf{lts\_cons\_trace} (line 8).  The former simply expresses that, from any state, 
	all \textsf{actions} of the \textsf{LTS} yield no target state (lines 3-5). Indeed, the
	function 
	\textsf{lts\_target\_state : LTS $\rightarrow$ lts\_state $\rightarrow$ message $\rightarrow$ lts\_state}
	is responsible computing the attained state from \textsf{q} with an \textsf{action} holding the message
	\textsf{m}. Thus, the constructor's conclusion is \textsf{LTS\_Trace A q LNil}, since the trace from \textsf{q}
	is empty.	The latter constructor however, demands the that we reach a state \textsf{q'} (line 11), and
	a trace from \textsf{q'} (line 12), in order to conclude \textsf{LTS\_Trace A q (LCons (Action m asgns) l)}.
		
	

	


	
	For the sake of clarity, Listing \ref{lst:ltstargetstate} 	
	
		\lstinputlisting[language=Coq, stepnumber=1, 
	                      caption={\textsf{lts\_target\_state} function definition}, 
	                      %firstnumber=19,
	                      label=lst:ltstargetstate]{listings/chapter6/lts\_target\_state.tex}		
	
	
	\noindent a



\section{Synchronization of LTSs, and traces}
\label{sec:pnet}




\section{Modelling GCM internals}
\label{sec:gcmpnets}



\section{Discussion}
\label{sec:behaviourdiscussion}


\chapbreak

	In this chapter we presented the mechanization of a behavioural 
semantics based on the execution trace of synchronized labelled transition systems. Further, we
exemplified its use in the context of \ac{GCM} applications.
	
	In the following chapter we discuss the works related with this thesis.



