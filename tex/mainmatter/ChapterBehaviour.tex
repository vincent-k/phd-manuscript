% Chapter 7

\chapter{Mechanized behavioural semantics} 
\label{chap:behaviour} 

\epigraph{\textit{“Yet, ..."}}{Nuno Gaspar}



\minitoc


\lhead{Chapter 7. \emph{Mechanized behavioural semantics}} % This is for the header on each page - perhaps a shortened title

%----------------------------------------------------------------------------------------

	This chapter discusses the mechanization, in the Coq proof assistant, of a behavioural 
semantics based on the execution trace of synchronized labelled transition systems. Further, we show
how it can be used in the context of \ac{GCM} applications.

	Section \ref{sec:pLTS} presents the mechanization of \ac{LTS} and
their traces. We show how we can synchronize several \ac{LTS} in Section \ref{sec:pnet}.
Then, we exemplify its use in the context of \ac{GCM} applications in Section \ref{sec:gcmpnets}.
For last, Section \ref{sec:behaviourdiscussion} discusses the final remarks about this
mechanization.


\section{Labelled transition systems, and traces}
\label{sec:pLTS}





\section{Synchronization of LTSs, and traces}
\label{sec:pnet}




\section{Modelling GCM internals}
\label{sec:gcmpnets}



\section{Discussion}
\label{sec:behaviourdiscussion}


\chapbreak

	In this chapter we presented the mechanization of a behavioural 
semantics based on the execution trace of synchronized labelled transition systems. Further, we
exemplified its use in the context of \ac{GCM} applications.
	
	In the following chapter we discuss the works related with this thesis.



