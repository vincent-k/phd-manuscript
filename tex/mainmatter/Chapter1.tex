% Chapter 1

\chapter{Introduction} 
\label{chap:intro} 
\epigraph{\textit{"Never permit a dichotomy to rule your life, a dichotomy in which you hate what you do so you can have pleasure in your spare time. Look for a situation in which your work will give you as much happiness as your spare time."}}{Pablo Picasso}

\minitoc

\lhead{Chapter 1. \emph{Introduction}} % This is for the header on each page - perhaps a shortened title

%----------------------------------------------------------------------------------------


		This thesis belongs to the domain of formal methods. In particular, we focus 
		their application on a specific methodology
	for the development of software: component-based engineering. 
	
		Let it be by means of automatic or interactive approaches, the goal of formal methods
	is to increase the confidence one can place on a software system. Throughout this thesis, 	
	we discuss the application of formal methods techniques in the context of 
	component-based engineering.     	
	


\section{Component-based software engineering}
\label{sec:cbse}

%By enforcing a strict separation between interface and implementation and by making software architecture explicit, 
%component-based programming can facilitate the implementation and maintenance of complex distributed software 
%systems

%abstract
	Among all programming paradigms, component-based engineering stands as one of the 
	most followed approaches for real world software development.  Its emphasis on a clean separation 
	of concerns makes it appealing for both industrial and research purposes.
	Furthermore, component-based systems are notorious for their capacity to address the inherent 
	challenges of today's software development.  These, promote modular designs,
	and therefore ease the burden of development and maintenance of applications. Moreover, portability 
	and re-usability of components are further benefits of this paradigm.

	%1 - core elements
	There are several component models proposed in the literature \cite{fractalSpec,BCDGHP:Telecom08}, 
	each with their own particularities
	(hierarchical/flat, static/reconfigurable, ...). Yet,
	their foundation is generally made of three main ingredients:
	\textit{components}, \textit{interfaces} and \textit{bindings}. A component can be seen as a 
	\textit{building block}, usually a piece of software code. An interface is an access point to/from 
	components. Finally, a binding is a connection established between components, 
	through their interfaces.	
	
	%2 - difference with OO 
	Indeed, component-based programming shares some resemblances with object-oriented programming.
	Their fundamental distinction lies at the fact that in the former paradigm, 
	communications are always	explicitly made through the component's interfaces, 
	therefore making all existing dependencies evident.

	%3 - reconfig 	
	Another facet of component-based systems concerns software evolution.	
	This methodology of developing software enables \textit{on-the-fly} structural 
	reconfigurations of the architecture of the application. The advantage of modifying the software 
	architecture at runtime comes from the need to cope with the plethora of situations 
	that may arise in a potentially massively distributed and heterogeneous system.  Indeed,
	the ability to restructure is also a key aspect in	the field of \textit{autonomic} computing 
	where software is expected to adapt itself.	However, this capacity comes with a price: 
	we no longer need only to care about functional concerns, but also  about structural ones.
	
	%4 - GCM	
		
	The widespread use of component models together with the interesting challenges posed by 
	reconfigurable component-based applications make it an exciting research
	topic for the formal methods community. Within our research group, we focus on the 
	\ac{GCM} \cite{BCDGHP:Telecom08} to address the intricacies of grid 
	and cloud computing. Details regarding its specification are discussed in Section \ref{sec:gcm}.



\section{Context --- The Spinnaker project}
\label{sec:spinnaker}


%\footnote{\url{http://www.spinnaker-rfid.com/}
%\url{http://www.inria.fr/en/centre/sophia/calendar/conference-of-christophe-loussert-tagsys-rfid}
%


	This thesis occurs in the context of the Spinnaker project\footnote{Project OSEO ISIS. \url{http://www.spinnaker-rfid.com/}}, 
	a collaborative project between INRIA and several industrial partners, where we intend to contribute for the widespread
	adoption of \ac{RFID}-based technology. To this end, our contribution comes with the design and implementation
	of a non-intrusive, flexible and reliable solution that can integrate itself with other already deployed
	systems.	Specifically, we developed \textsc{The HyperManager}, a general purpose monitoring
	application with autonomic features. This was built using GCM/ProActive\footnote{\url{http://proactive.activeeon.com/index.php}} --- a Java middleware for parallel
	and distributed programming that follows the principles of the \ac{GCM}. For the purposes
	of this project, it had the goal to monitor the \textsf{E-Connectware}\footnote{\url{http://www.tagsysrfid.com/Products-Services/RFID-Middleware}}  
	(\textsf{ECW}) framework in a loosely coupled manner.



%\section{Motivation}
%\label{sec:motivation}


%	Furthermore, the expressiveness found in interactive theorem provers allows us to reason about properties
%	that cannot be expressed in a model-checker. 
	
%	 Moreover, it is in our plans to
%	extend the expressiveness of the GCM ADL by including the possibility to define architectures 
%	that would not necessarily be an explicit representation of its structure, but parametrized. Reasoning 
%	on such \textit{parametrized} ADLs would feel natural in a system like a proof assistant, as
%	it essentially boils down to reason inductively on the parameters.


\section{Contributions}
\label{sec:contrib}

	
		In this thesis, we contribute to the state of the art in the domain
	of formal methods and (distributed) component-based systems.
		
	Our first contribution is an industrial
	case study on the behavioural specification and verification of a reconfigurable
	distributed application. This promotes the use of formal methods in an industrial context,
	but also puts in evidence the necessity for alternative and/or complementary approaches 
	in order to better address such undertakings.		
		
	Our second contribution is a framework, developed with the Coq proof assistant,
	for reasoning on software architectures: \textsc{Mefresa}. This encompasses
	the mechanization of the \ac{GCM} specification, and the means to
	reason about reconfigurable \ac{GCM} architectures. Further, we address
	behavioural concerns by formalizing a semantics based on execution traces of
	synchronized transition systems.
	Overall, it provides the first steps towards a complete specification and verification
	platform addressing both architectural and behavioural properties.
						
	Finally, our third contribution is a new \ac{ADL}, denominated \textsc{Painless},
	and its proof-of-concept integration with the ProActive middleware. 
	\textsc{Painless} allows to specify parametrized \ac{GCM} architectures, along with
	their structural reconfigurations, in a declarative-like language. Compliance
	with the \ac{GCM} specification is evaluated by certified	 functional code extracted from
	\textsc{Mefresa}. This permits the safe deployment and
	reconfiguration of \ac{GCM} applications.
			
	The following publications resulted from this thesis.
	
	\begin{itemize}
		\item Nuno Gaspar and Eric Madelaine. \textit{Fractal \'a la Coq}.
		Conf\'erence en Ing\'enierie du Logiciel, Rennes, France, June 2012.
		
		\item Nuno Gaspar, Ludovic Henrio and Eric Madelaine. 
		\textit{Bringing Coq into the World of GCM Distributed Applications}.
		International Journal of Parallel Programming, pp. 1-20, 2013.
		
		\item Nuno Gaspar, Ludovic Henrio and Eric Madelaine.
		\textit{Formally Reasoning on a Reconfigurable Component-Based 
		System --- A Case Study for the Industrial World}.
		International Symposium on Formal Aspects of Component 
		Software (FACS'2013), October 2013.  
		
	\item  Nuno Gaspar, Ludovic Henrio and Eric Madelaine.
	\textit{Painless support for the static and runtime verification of component-based applications. 
	(submitted)}		
		
	\end{itemize}


	Moreover, a talk
	entitled \textit{Formal Reasoning on Component-Based Reconfigurable Applications},
	was given at the student session of the 40$^{th}$ ACM SIGPLAN-SIGACT Symposium on 
	Principles of Programming Languages (POPL'2013).


\section{Organisation of this thesis}
\label{sec:struct}

	The remaining of this thesis is organized as follows.


	\begin{itemize}
		\item Chapter \ref{chap:preli} gives an overview of the \textbf{technical background} required for the understanding 
			      of this thesis. Pointers for relevant literature are indicated for the reader desiring further
				  details.				
				
				  
		\item Chapter \ref{chap:hyper} discusses an \textbf{industrial case study} on the formal specification and verification
			     of a reconfigurable GCM/ProActive application. This is achieved by formally defining its behavioural
				 semantics and check it against the desired properties by means of model-checking techniques. 
			     		
		
		\item Chapter \ref{chap:mefresa} presents \textsc{Mefresa}, a \textbf{framework for the reasoning on software architectures}.
				  It is tailored for the \ac{GCM}, and developed using the Coq proof assistant. 
				  
				  
		\item Chapter \ref{chap:extraction} shows how we leverage the \textsc{Mefresa} framework and Coq's certified extraction 
		         mechanism	to provide an \textbf{extension to the GCM/ProActive middleware}. We propose a 
		         new approach for the specification of reconfigurable \ac{GCM} architectures.		
		
		\item Chapter \ref{chap:behaviour} addresses the mechanization of a \textbf{behavioural semantics using the Coq proof assistant}. We show how we can interactively reason about transition
		systems.
			     Further, we illustrate its use in the context of the \ac{GCM}.
		
		
		\item Chapter \ref{chap:related} discusses relevant \textbf{related work} w.r.t. to the main contributions of this thesis.			
				  	
					

		\item For last, chapter \ref{chap:conclusion} discusses the \textbf{final conclusions} about this thesis, and 
		         indicates perspectives for future work and improvements.
		
		
	\end{itemize}



%\begin{flushright}
%Guide written by ---\\
%Sunil Patel: \href{http://www.sunilpatel.co.uk}{www.sunilpatel.co.uk}
%\end{flushright}
