% Chapter 6

\chapter{Related work} 
\label{chap:related} 


\epigraph{\textit{“If you know the enemy and know yourself you need 
                             not fear the results of a hundred battles."}}{Sun Tzu}

\minitoc

\lhead{Chapter 6. \emph{Related Work}} % This is for the header on each page - perhaps a shortened title

%----------------------------------------------------------------------------------------

		This Chapter discusses the works related with this thesis. Section \ref{sec:relhyper} places the industrial
	case study discussed in Chapter \ref{chap:hyper} in the context of existing literature regarding behavioural
	specification and verification. Section \ref{sec:relmefresa} discusses other undertakings concerning the formalization
	of component models using proof assistants or more lightweight specification platforms. For last, 
	Section \ref{sec:relpainless} compares \textsc{Painless} and its integration with the ProActive middleware
	with other \ac{ADL}s and related tools.
	

\section{On the HyperManager use case}
\label{sec:relhyper}

	%byzantine fault : disse que queria reconfig.
	%use case cadp site...
	%BIP framework? ou na secao mefresa?

	\cite{BHHM:FACS11} madelaine case study
	
	
	Compositional Verification for Component-Based Systems and Application
	http://www-verimag.imag.fr/~sifakis/atva2008.pdf
	\cite{BBN+10} 


	FACS:
	  The maturity attained by the CADP toolbox 
		made it a reference tool among the formal methods community.  Several case studies have been published, 
		namely industrial ones addressing other goals than verification. For instance, in \cite{conf/cav/CosteHLS09}
	Coste et. al. discuss performance evaluation for systems and networks on chips.			
	 More closely related with our work we must refer the experiments presented in 
	 \cite{conf/dais/CornejoGMP01}. A dynamic reconfiguration protocol is specified and model-checked,
	 however their focus is on the reconfiguration protocol itself rather than reconfigurable applications.
		 
		  	
		Indeed, many works can be found in the literature embracing a behavioural semantics
		approach for the specification and verification of distributed systems. Yet, 
		literature addressing the aspects of reconfigurable applications remains 
		scarce. 

		
 Nevertheless, we must cite the work around
	    %BIP
	    BIP (Behaviour, Interaction, Priority) \cite{BBB+11a} --- a framework 
	 encompassing rigorous design principles. It allows the description of the 
	 coordination between components
	 in a layered way. Moreover, it has the particularity of also permitting the 
	 generation of code from its models. Yet, structural reconfigurations are not supported. 
	 	
	%CHARMY	
	Another rather different approach that we must refer is the one followed by tools 
	specifically tailored for architectural specifications. For instance, in \cite{p.2005-1} Inverardi et. al. 
	discusses \textsf{CHARMY}, a framework for designing and validating architectural 
	specifications. It offers a full featured graphical interface with the goal of being more 
	\textit{user friendly} in an industrial context. Still,
	architectural specifications remain of static nature. 
	
	
	
		Looking at the interactive theorem proving arena we can also find some
	related material. In %coq stuff
	\cite{Boyer:2013:RRC:2486788.2486791} Boyer et. al. propose a reconfiguration protocol
	and prove its correctness in the Coq Proof Assistant \cite{09thecoq}. This work however, focuses on the protocol itself, 
	and not in the behaviour of a reconfigurable application. 
   





\section{On the Mefresa framework}
\label{sec:relmefresa}


%creol

	%    henrio.
 %   Nev- ertheless, we must cite the work from Henrio et al. [12] on a framework for reasoning on the behaviour of GCM component composition mechanized in Isabelle/HOL [18]. Our approach is still considerably different in that we pro- vide a semantics for an operation language for the building and reconfiguration of GCM architectures.
 
 Many approaches regarding the formalization of component models can be found in the literature. Yet,
	to the best of our knowledge this is the first work aiming at providing a mechanized framework
    that applies the \textit{correct-by-construction} paradigm to the world of component-based engineering.
	Nevertheless, we must cite the work from Henrio et al. \cite{HKK:FMCO09} on a framework for reasoning 
	on the behaviour of GCM component composition mechanized in Isabelle/HOL \cite{Nipkow-Paulson-Wenzel:2002}. 
	Our approach is still considerably different
	in that we provide a semantics for an \textit{operation} language for the building 
	and reconfiguration of GCM architectures.
	
	%work of Coq from INRIA Sardes	: http://perso.ens-lyon.fr/damien.pous/rrca/ http://2013.icse-conferences.org/
	
	
	%CREOL
	Another work involving the use of a proof assistant is the work by Johnsen et. al. 
	on the \textsf{Creol} framework \cite{johnsen06tcs}. \textsf{Creol} focuses on
	the reasoning of distributed systems by providing a high-level object oriented
	modelling language. Its operational semantics are defined in the rewriting logic tool 
	Maude \cite{Maude2:03}. This work however does not contemplate architectural 
	reconfigurations and follows a methodology in the style of a Hoare Logic. 	
		
    

   	
	%most stuff is "just" model checking
	%invited speaker conf. http://www.di.univaq.it/inverard/
	%http://www.di.univaq.it/charmy/
	Indeed, while the use of proof assistants is gaining notoriety, model-checking is still
	the \textit{de facto} formal approach for both industrial and academic undertakings.
	Usually, some form of state machine model is used for the design of the intended
	system,  and then a carefully chosen model-checker as back-end is employed as
	a decision procedure. %This is for instance the methology followed in our Vercors
	%platform \cite{Barros:2007:MDC:1269985.1270075}. 
	For instance, in \cite{p.2005-1} Inverardi et. al. 
	discusses \textsf{CHARMY}, a framework for designing and validating architectural 
	specifications. It offers a full featured graphical interface with the goal of being more 
	\textit{user friendly} in an industrial scenario. Still,
	architectural specifications remain of static nature. 
	The BIP (Behaviour, Interaction, Priority) 
	framework \cite{BBB+11a} formalizes and specifies component interactions.
	It has the particularity of also permitting the generation of code from its models.
	

	%formal spec alloy  -> they recommend Coq
   	Moreover, a formal specification of the Fractal Component Model has 
   	been proposed in the  Alloy specification language \cite{MERLE:2008:INRIA-00338987:1}. 
	This work proves the consistency of a (set-theoretic) model of Fractal applications. 	 
	Their goal was to clarify the inherent ambiguities of the informal specification presented
  	in \cite{fractalSpec}. Their specification however, is constrained by the first-order relational logic nature 
   	of the Alloy Analyzer. In fact, they point to the use of the Coq Proof Assistant in order 
   	to overcome this limitation.	
	
	
	%era PC ano passado
	%Let us also mention previous work by Aldrich on the use of a type system at the source code level to 
	%enforce architectural constraints \cite{Aldrich:2008:UTE:1343599.1344114}. This is a rather different
	%approach than ours as it is a direct extension of the Java programming language with type annotations
	%that are statically checked. Allowing for flexible and dynamic designs is within the goals of the project,	
	%but structural reconfigurations are not supported.


	%\paragraph{}
	%see JNO 

	%in the spirit of dynamic ADLs
	%\paragraph{}
	%may want to refer dynamic ADLs ?

	%fpath &fscript
	For last, from a more engineering point of view, we can refer the work around FPath and FScript \cite{journals/adt/DavidLLC09}.
	FPath is a domain-specific language for the navigation and querying of Fractal architectures. FScript embeds the FPath
	language and acts as a scripting language for the specification of reconfigurations strategies. The main goal of this
	work however is to alleviate the need to interact with the low-level API. Further, reliability of these reconfigurations
	is ensured by run-time checking, while we are more concerned on providing guarantees statically.
	
	%\paragraph{}
	%Extension to previous work \cite{gaspar:hal-00725291}


\section{On Painless}
\label{sec:relpainless}

	Several proposals for \ac{ADL}s can be found in the literature, each with their own 
	particularities and attempting to 
	address their specific application domain concerns. 
	
	%http://www-systems.cs.st-andrews.ac.uk/wiki/ArchWare/
	The work around the ArchWare \ac{ADL} \cite{DBLP:journals/corr/abs-1006-4829} 
	is of particular interest. They claim that 
	\textit{"software that cannot change is condemned to atrophy"}	
	and introduce the concept of an \textit{active software architecture} in order
	to address this challenge. Based on the higher-order $\pi$-calculus, it provides
	constructs for specifying control flow, communication and dynamic topology.
	Unlike \textsc{Painless}, its syntax exhibits an imperative style 
	and type inference is not supported, thus not promoting concise specifications. 
	
	Nevertheless, it is sufficiently rich to provide executable specifications of active software
	architectures. Moreover, user-defined constraints are supported through the ArchWare 
	Architecture Analysis Language. Yet, their focus is more aimed at the specification
	and analysis of the \ac{ADL}, rather than actual application execution and deployment.
	In our work, the user solely defines the architecture of its application, structural constraints
	are implicit: they are within the mechanized \ac{GCM} specification.	Further, our tool support
	is tightly coupled with the ProActive middleware.
	
		
	%ref Archery
		Also from the realm of process algebras, Archery \cite{conf/facs2/SanchezBR11} is a modelling
	language for software architectural patterns. It is composed by a core language and two extensions:
	\textsf{Archery-Core}, \textsf{Archery-Script} and \textsf{Archery-Structural-Constraint}.
	These permit the specification of structural and behavioural dimensions of architectures,
	the definition of scripts for  reconfiguration, and the formulation of structural constraints,
	respectively. Moreover, a bigraphical semantics is defined for Archery specifications. This grants
	the reduction of the constraint satisfaction verification to a type-checking problem. However,
	this process is not guarantee to be automatic, and type-checking decidability remains as future work.
	
	
	%%check refs....
	%falta darwin..
				
		
	%logic based. http://ce-resd.facom.ufms.br/sbrc/1994/p10.pdf
	As expected, logic is also a natural formalism of choice. For instance, Gerel \cite{endler1992}
	is a generic reconfiguration language including powerful query constructs based on first-order logic.
	Further, its reconfiguration procedures may contain preconditions \textit{\`a la} 
	Hoare Logic \cite{Hoare69anaxiomatic}. These are evaluated by brute force. It is unclear
	how they cope with the inherent undecidability of such task. 
	%paper runtime verif. http://hal.archives-ouvertes.fr/docs/00/64/23/45/PDF/facs2011_preproceedings_9.pdf
	%favs 2011
	%temporal logic to FScript ... undecidable problem in conclusion...


		%graph based:
	Furthermore, the use of a graph-based formalism for the specification of software architectures
	is also rather popular. For instance, in \cite{Bruni:2008:GDA:1424922.1424928} two graph-based
	approaches and related tool support are compared. Another interesting aspect of this work
	regards the use of the Alloy Analyzer \cite{Jackson:2002:ALO:505145.505149}, 
	and Maude \cite{Clavel:2002:MSP:633559.633563} as a means to show the approaches' feasibility. 
		

		More recently, 	Di Cosmo et. al. defined the Aeolus component model \cite{conf/sefm/CosmoZZ12}. Their focus
	is on the automation of cloud-based applications deployment scenarios. Nevertheless, their 
    proposal is still loosely inspired by the Fractal component model \cite{fractalSpec} whose
	most peculiar characteristics are its hierarchical composition nature and reconfiguration capabilities.
	However, while both approaches permit architectural reconfigurations at runtime, 
    its specification is not supported by their ADL, it solely contemplates deployment related
    aspects. Moreover, support for
	parametrized specifications is also not covered, forcing the software architect to explicitly 	
	define the application's structure.    
	
		Regarding Fractal,  it is also worth noticing that they try to overcome the lack of
	support for reconfiguration specification	 through 
	Fscript \cite{DAVID:2008:HAL-00468474:1}. Fscript embeds FPath --- a DSL for navigation and querying of
	Fractal architectures --- and acts as a scripting language for reconfiguration strategies. These
	are not evaluated for their validity. Nevertheless, system consistency is ensured by the use of
	\textit{transactions}: a violating reconfiguration is \textit{rolled back}.
	Furthermore, an interesting feature is the support for application-specific architectural
	invariants.


	%ref survey 2/5
	%	scalability / types.. logic etc
	%question "Given system x, what happens when change y occurs?"
	%para tese devo colocar me em todas as tabelas referidas..
		To conclude, let us mention a survey regarding the approaches for the specification of 
	dynamic software architectures \cite{Bradbury:2004:SSD:1075405.1075411}. These are 
	classified into four categories: graph-based, process algebra-based, 
	logic-based, and other. To this end, we place ourselves
	in the \textit{other} category. Indeed, while formal logic is at the foundation of 
	our approach, all the machinery involving requirement/constraint checking is 
	implicit and done automatically by our Coq development. The software architect 
	solely uses a language with a declarative trait. %Another important factor referred by 
	%this survey concerns scalability. In general, a decentralized or distributed approach
	%is more likely to scale. This is the case for \textsc{Painless}, as	
	
	
	
		
	
		
	
	
	
	 %oleksandra
    
    
    %Indeed, its structural information remains of static nature, therefore 
	%not revealing the full picture regarding an application topology. 	





