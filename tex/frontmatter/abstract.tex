%\selectlanguage{french}
%\begin{abstract}

%GCM

%\end{abstract}

\chapter*{\centering R\'esum\'e}

Cette thèse appartient au domaine des méthodes formelles. 
Que ce soit par le biais d'approches automatiques ou interactives, l'objectif des méthodes formelles 
est d'accroître la confiance que l'on peut placer sur les propriétés d'un système. 
Dans cette thèse, nous concentrons 
leur application sur une méthodologie spécifique 
pour le développement de logiciels: l'ingénierie à base de composants. 


Parmi tous les paradigmes de programmation, de l'ingénierie à base de composants se présente comme une 
des approches les plus suivies pour le développement de logiciels dans le monde réel. Elle met l'accent 
sur une séparation nette des préoccupations, attrayant ainsi pour des fins industriels 
et de recherche. En outre, elle permet des reconfigurations structurelle 
de l'architecture des applications. L'avantage de la modification de l'architecture du logiciel 
vient de la nécessité de faire face à la multitude de situations 
qui peuvent survenir dans un système potentiellement massivement distribué et hétérogène. 
À cette fin, le \textit{Grid Component Model} (GCM) suis cette approche en fournissant touts  
les moyens pour définir, composer, et dynamiquement reconfigurer les applications distribuées 
à base de composants. 


Dans cette thèse, nous abordons la spécification formelle, la vérification et le déploiement
 d'applications GCM reconfigurables et distribués. Notre première contribution est un cas d'étude 
industriel 
 sur la spécification comportementale et la vérification d'une application distribuée et 
 reconfigurable: \textsc{The HyperManager}. 
Cela favorise l'utilisation des méthodes formelles dans un contexte industriel, 
     mais met également en évidence la nécessité d'approches alternatives et/ou complémentaires 
afin de mieux répondre à ces entreprises. 

Notre deuxième contribution est une plate-forme, élaboré avec l'assistant de preuve Coq, 
pour le raisonnement sur ​​les architectures logicielles: \textsc {Mefresa} . Cela comprend 
la mécanisation de la spécification du GCM, et les moyens pour 
raisonner sur les architectures reconfigurables GCM. En outre, nous adressons 
les aspects comportementaux en formalisant une sémantique basée sur les traces d'exécution de 
systèmes de transitions synchronisées. 
Dans l'ensemble, elle fournit les premiers pas vers une plate-forme complète pour la spécification 
et la vérification des caractéristiques structurelle et comportementale. 

Enfin, notre troisième contribution est un nouveau langage de description d'architecture (ADL): 
\textsc{Painless}. En outre, nous discutons son intégration avec ProActive --- 
un intergiciel Java pour la programmation concurrente et distribuée, 
et l'implantation référence du GCM. 
\textsc{Painless} permet de spécifier des architectures de GCM paramétrées, avec 
leurs reconfigurations structurelle, dans un langage déclaratif. Conformité 
avec la spécification de GCM est évaluée par le code fonctionnel certifié extrait de 
\textsc{Mefresa} . Ceci permet le déploiement et
reconfigurations sûre des applications du GCM.





%\selectlanguage{english}
%\begin{abstract}

\chapter*{\centering Abstract}
\addcontentsline{toc}{chapter}{Abstract}

		This thesis belongs to the domain of formal methods.  
	Let it be by means of automatic or interactive approaches, the goal of formal methods
	is to increase the confidence one can place on a system's properties.
	In this thesis, we focus 
		their application on a specific methodology
	for the development of software: component-based engineering.


		Among all programming paradigms, component-based engineering stands as one
	of the most followed approaches for real world software development. Its emphasis
	on a clean separation of concerns makes it appealing for both industrial
	and research purposes. Further, it enables \textit{on-the-fly} structural 
	reconfigurations of the architecture of the application. The advantage of modifying the software 
	architecture at runtime comes from the need to cope with the plethora of situations 
	that may arise in a potentially massively distributed and heterogeneous system.	
	To this end, the Grid Component Model (GCM) endorses this approach by providing all 
	the means to define, compose and dynamically reconfigure component-based distributed applications.
		
	
		In this thesis we address the formal specification, verification and deployment of
	distributed and reconfigurable GCM applications. Our first contribution is an industrial
	case study on the behavioural specification and verification of a reconfigurable
	distributed application: \textsc{The HyperManager}. 
	This promotes the use of formal methods in an industrial context,
    but also puts in evidence the necessity for alternative and/or complementary approaches 
	in order to better address such undertakings.		
		
	Our second contribution is a framework, developed with the Coq proof assistant,
	for reasoning on software architectures: \textsc{Mefresa}. This encompasses
	the mechanization of the GCM specification, and the means to
	reason about reconfigurable GCM architectures. Further, we address
	behavioural concerns by formalizing a semantics based on execution traces of
	synchronized transition systems.
	Overall, it provides the first steps towards a complete specification and verification
	platform addressing both architectural and behavioural properties.
						
	Finally, our third contribution is a new Architecture Description Language (ADL), 
	denominated \textsc{Painless}. Further, we discuss its proof-of-concept integration 
	with ProActive --- a Java middleware for concurrent and distributed programming,
	and the \textit{de facto} reference implementation of the GCM. 
	\textsc{Painless} allows to specify parametrized GCM architectures, along with
	their structural reconfigurations, in a declarative-like language. Compliance
	with the GCM specification is evaluated by certified	 functional code extracted from
	\textsc{Mefresa}. This permits the safe deployment and
	reconfiguration of GCM applications.
	


%\end{abstract}


\chapter*{\centering Resumo}

Esta tese pertence à disciplina de m\'etodos formais. 
Que seja por meio de abordagens automáticas ou interativas, o objectivo dos métodos formais 
é aumentar a confiança se pode colocar sobre as propriedades de um sistema. 
Nesta tese, focamos a 
sua aplicação numa metodologia específica 
para o desenvolvimento de \textit{software}: engenharia baseada em componentes. 


De entre todos os paradigmas de programação, a engenharia baseada em componentes destaca-se como uma 
das abordagens mais usada para o desenvolvimento de software do mundo real. A sua ênfase 
numa clara separação de functionalidades e controlo, torna-a atraente para fins industriais 
e de investigação. Além disso, permite reconfigurações estruturais 
 da arquitetura da aplicação. A vantagem de alterar a topologia da aplicação
 em tempo de execução vem da necessidade de lidar com a multiplicidade de situações 
que podem surgir em sistemas distribuídos. 
O \textit{Grid Component Model} (GCM) segue esta abordagem, fornecendo todos 
os meios para definir, compor e reconfigurar dinamicamente aplicações distribuídas baseadas em componentes. 


Nesta tese abordamos a especificação formal, verificação e implementação de 
aplicações GCM distribuídas e reconfiguráveis​​. A nossa primeira contribuição é um caso de estudo industrial 
sobre a especificação comportamental e verificação de uma 
aplicação distribuída e reconfigurável: \textsc{The HyperManager}. 
Isto promove o uso de métodos formais num contexto industrial, 
mas também põe em evidência a necessidade de abordagens alternativas e/ou complementares 
para este tipo de tarefas. 

A nossa segunda contribuição é um modelo desenvolvido com o assistente de prova Coq, 
para o raciocínio em arquiteturas de software: \textsc{Mefresa} . Este, engloba 
a mecanização da especificação GCM, e os meios para 
raciocinar sobre arquiteturas reconfiguráveis ​​GCM. Além disso, abordamos 
aspectos comportamentais através da formalização de uma semântica com base em traços de execução de 
sistemas de transição sincronizados. 
Em suma, fornecemos os primeiros passos para uma plataforma de especificação e verificação 
abordando propriedades de arquitetura e comportamentais. 

Por fim, a nossa terceira contribuição é uma nova \textit{Architecture Description Language} (ADL), 
denominada \textsc{Painless}. Além disso, discute-se a sua integração com \textsf{ProActive} 
--- um middleware Java para programação concorrente e distribuída, 
e a implementação referência do GCM. 
\textsf{Painless} permite especificar arquiteturas GCM parametrizadas, juntamente com 
suas reconfigurações estruturais, em uma linguagem declarativa. Conformidade 
com a especificação GCM é avaliada pelo código certificado extraído de \textsc{Mefresa}. 
Isto, permite o lançamento e reconfiguração segura das aplicações GCM.


\selectlanguage{english}
